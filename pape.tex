\documentclass[10pt,a4j,twocolumn]{jarticle}
\pagestyle{empty}
\usepackage[dvips]{color}
\usepackage[dvips]{graphicx}
\usepackage[dvipdfm,margin=3em]{geometry}
\usepackage{amsmath}
\usepackage{multirow}
\usepackage{txfonts}
\usepackage{listings,jlisting}
\usepackage{url}
\renewcommand{\lstlistingname}{プログラム}
\lstset{
	basicstyle=\ttfamily\footnotesize,%
	commentstyle=\textit,%
	classoffset=1,%
	tabsize=4,%
	keywordstyle=\bfseries,%
	frame=tb,framesep=5pt,%
	showstringspaces=false,%
	breaklines=true,%
	%numbers=left,stepnumber=1,numberstyle=\footnotesize%
}%
\title{自分を機械学習してみよう}
\author{仮}
\date{\hfill}
\begin{document}
	\maketitle
	\thispagestyle{empty}
	\section{初めに}
	近年機械学習という分野は、目まぐるしい進歩を遂げ様々な用途に使われてきている。
	人工知能の開発に始まり、ビジネスでの用途や様々な予測など枚挙に遑がない。
	本書では、その機械学習をフィーチャーし全く役に立たない予測を適当に行おうという企画である。
	ちなみに筆者の機械学習についての知識は、果てしなく0に近いということを念頭に置いて読むことを推奨する。

	\section{本書について}
	本書では、身体情報を収集し、そのデータから様々な予測を行っていく。
	身体情報の収集には、Microsoftから発売されているMicrosoft Band(以下Band)を使用する。
	機械学習のプラットフォームは、Microsoft Azureで提供されているAzure MLを使用する。
	また、Bandは、単体でWebにつながる手段を持っていないためデータの収集、アップロードには、
	マウスコンピュータから発売されているWindows Phone 8.1搭載スマートフォンであるMADOSMAを使用する。

	\section{本書で使用するデータ}
	本書では、主にBandから収集できるデータを使用する。そこで、以下に示すデータを主に使用していく。
	\begin{itemize}
		\item 心拍数
		\item 一日の歩数
		\item 一日の移動距離
		\item 一日の睡眠時間
		\item 一日の消費カロリー
	\end{itemize}
	これにプラスして仕事の勤務時間をデータとして使用することにより様々なものを適当に予想していく。

	\section{本書で使用するデバイス}

	\subsection{Bandについて}
	Bandとは、Microsoftが製造、販売するウェアラブルデバイスであり、主にフィットネスにフォーカスしたデバイスである。
	Apple WatchやAndroid Wearと異なる点として
	\begin{itemize}
		\item Windows Phone
		\item iOS
		\item Android
	\end{itemize}
	と主要なスマートフォンOSに対応し、専用アプリを使用することによって睡眠時間や心拍数などの身体情報の収集に加え、
	ワークアウト時間の計測や消費カロリー計算などの計測が行える。

	このセンサーは、専用のSDKを使用することにより簡単に自身のアプリに組み込むことも可能である。
	また、ウェアラブルデバイスらしくメールやSNSなどの通知に加え、電話の通知が取得できる。
	ちなみにWindows Phoneと連動している場合に限りCortanaを使用した音声操作などの機能に対応している。
	2015年8月現在アメリカのみので販売であるが、アメリカ国内では、スターバックスの電子決済サービスにも対応しているといった特徴を持つ。

	\subsection{MADOSMAについて}
	MADOSMAは、2015年6月にIS-12Tから約4年ぶりとなるWindows Phoneの日本国内販売モデルである。
	発売元は、マウスコンピュータ。
	5インチの大型ディスプレイに加え、128gと類似サイズの他社スマートフォンと比べて非常に軽量であるという特徴を持つ。
	職人の技が光る素晴らしい出来のスマートフォンである。

	ちなみに例に漏れず電源ボタンと音量ボタンの同時押しでスクリーンショット撮影機能が搭載されているが電源ボタンと音量ボタンの位置が非常に
	微妙であるためよく押し間違えてスクリーンショットを撮影してしまう弱点がある。
	電車内で操作を誤ると社会的に死ぬ可能性をはらんだスマートフォンである。

	\section{本書で使用するプラットフォームについて}
	\subsection{Microsoft Azure}
	Microsoft Azure(以下Azure)は、Microsoftが提供するクラウドプラットフォームである。
	登場当初は、主にPaas(Platform as a Service)を生業としたクラウドであったが、2015年現在では、IaaS(Infrastructure as a Service)
	やBaaS(Backend as a Service)といったものからAzure MLのような機械学習プラットフォームまで幅広く提供するため基本的になんでも可能である。

	また、Azureは、エンタープライズ向けのサービスに力を入れているため、エンタープライズ向けにかなり手厚いサポートやデータの多重冗長、
	ジオ冗長など企業がビジネスで運用することを想定したサービスである。

	Azureのデータセンターは、2015年8月現在稼働中のものが19リージョン存在し、これは、AmazonとGoogleの持つデータセンターの合計よりも多い数となる。
	また、一般向けには、カナダとインドに建設予定のリージョンが存在している。
	一般向け以外であると、政府専用のリージョンが用意されることになっていたりと、一般からエンタープライズ、果ては、政府向けまで対応するクラウドサービスである。

	\subsection{Azure ML}
	Azure MLは、Azure上で提供されている機械学習プラットフォームである。
	データを対応する形に整形してアップロードし、あとは、GUIで学習に使用するアルゴリズムやデータの選定などを
	行う手法を選択して組み合わせることによって手軽に本格的な機械学習が行えるプラットフォームとなっている。

	機械学習自体には、それ相応の知識が必要であるが、環境の手間を省いたり、学習した情報をWebAPIとして取得する手段を
	簡単に公開できたりと機械学習の本質以外の部分を省くことができるという点で非常に有用な道具であるといえる。

	\section{本書で使用するシステムの構成}
	以下に本書でデータ収集を行う際に使用するシステムの構成について示す。


	\section{まとめ}

	\begin{thebibliography}{9}
		\bibitem{bib:bib001} 名前 : 題名, 出版社, 西暦
	\end{thebibliography}


\end{document}
